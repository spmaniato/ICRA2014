...

\begin{myExample}\label{Ex:mailman3} Mailman (revisited)
	Give all details of the spec ... \textbf{Same as demo or multiple letters but other simplifications?}
\end{myExample}

\begin{algorithm}
	\textbf{Mission specification:} Robotic Mailman\\
	{\small
	\texttt{Group Letters is empty}\\
	\texttt{Group CarryingLetters is empty}\\
	\texttt{Group Offices is empty}\\
	\texttt{CarryingLetters correspond to Letter}\\
	\texttt{Offices correspond to Letter}\\
	\texttt{DeliverLetter correspond to Letter}\\
	\texttt{each CarryingLetter is set on PickUp and the corresponding Letter and reset on the corresponding DeliverLetter}\\
	
	\texttt{if you are not activating any CarryingLetter then "patrol"?.}\\
	\texttt{...}\\
	}
	\textbf{Exploration Settings:}\\
	{\small
	\texttt{...} 
	}
\end{algorithm}

\begin{myExample}\label{Ex:planetxplore} Autonomous Planetary Exploration\\
	Consider now the more futuristic scenario of autonomous planetary exploration using rovers. In order to lessen the dependency of the rover from the engineers on Earth, the rover will be given a mission specification that it should carry out autonomously. However, the need to redefine the mission will arise as the rover discovers interesting elements of its environment, such as new types of rock and regions interest. In these cases, the engineers and scientists on Earth should be able to extend the rover's mission specification without sacrificing correctness or rewriting the specification from scratch.
	
	Specifics: exploration, new stuff sensor, new requests.
\end{myExample}

\begin{algorithm}
	\textbf{Mission specification:} Autonomous Planetary Exploration\\
	{\small
	\texttt{...}\\
	}
	\textbf{Exploration Settings:}\\
	{\small
	\texttt{...} 
	}
\end{algorithm}

% END