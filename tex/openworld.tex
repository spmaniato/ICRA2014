\subsection{Defining Tasks over Open Worlds}

First of all, let us define the notion of open world in the current context.

\begin{myDefinition}
	\textbf{(Open World).} The union of $\mathcal{X}$ and $\mathcal{Y}$, the sets of environment and robot proposition respectively, defines the state-space. If the state-space is allowed to increase in size during execution, then we say that the robot operates under the open world assumption.
\end{myDefinition}

In real-world applications, such as search and rescue scenarios, there is a need to define a mission before the robot has obtained full knowledge of the world. In other cases, the state-space may be too large for (i) the user to enumerate every possible variation upfront, and (ii) the synthesis algorithm to deal with. To clarify the latter cases, consider the following mission:

\begin{myExample}\label{Ex:mission1}
	A robotic mailman operates within a school building. It is tasked with collecting letters from a mailroom, and delivering them to the offices of the corresponding recipients. However, the robot's mission specification does not have to account for every possible recipient. Using the open world abstractions introduced in Section \ref{abstractions}, we can describe one of the robot's objective, in Structured English, as follows: 
\texttt{if you are sensing any Letter then do PickUp and go to corresponding Office}, where \texttt{Letter} and \texttt{Office} are groups of environment and location (robot) propositions, respectively.
%\texttt{each CarryLetter is set on PickupLetter and the corresponding Letter and reset on DeliverLetter}, where \texttt{CarryLetter} and \texttt{Letter} are groups of robot and environment propositions, respectively, and \texttt{PickupLetter} is a robot proposition that abstracts an actuator (e.g. a hand or gripper). 
The information on the letters themselves provides the correspondence between each letter and the location of the recipient's office. Then, during execution, as new letters are received, those groups are populated with propositions, the robot's discrete abstraction is updated, and the mission specification is translated into LTL formulas accordingly.
\end{myExample}

...

\subsection{Exploring the Physical Workspace}

In the robotics domain, more often than not, the state-space has a physical subset: the workspace that the robot operates in. When defining tasks using the open world abstractions that were introduced in Section \ref{abstractions}, one needs to account for the fact that the robot could simply choose to ignore those tasks. In other worlds, there has to be a mechanism that forces the robot to expand its physical workspace, in order to discover aspects of its mission specification.

We dedicate a portion of the mission specification to this ``exploration" task. It is worth noting that (i) the approach is still valid in the case that the robot has full knowledge of its workspace, but not its entire state-space, and (ii) that, since exploration is part of the specification, the user can define how the robot will deal with the open world, and under which conditions it will perform actions such as exploration or re-planning.

... frontier, BFS/DFS, memory, ...

\subsection{Augmenting the State-Space} % Re-synthesis (not local yet)

\begin{itemize}
	\item add to / remove from
	\item re-synthesis
	\item Not just LTL anymore (add to / remove from Group). First-order logic?
	\item ...
\end{itemize}