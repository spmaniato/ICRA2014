\subsection{Defining Tasks over Open Worlds}

Use new abstractions to define tasks for open worlds \ldots

\begin{myExample}\label{Ex:mailman2} Robotic Mailman (revisited)\\
	Using the open world abstractions introduced in Section \ref{abstractions}, we can describe one of the robot's objective, in Structured English, as follows: 
\texttt{if you are sensing any Letter then do PickUp and go to corresponding Office}, where \texttt{Letter} and \texttt{Office} are groups of environment and location (robot) propositions, respectively.
%\texttt{each CarryLetter is set on PickupLetter and the corresponding Letter and reset on DeliverLetter}, where \texttt{CarryLetter} and \texttt{Letter} are groups of robot and environment propositions, respectively, and \texttt{PickupLetter} is a robot proposition that abstracts an actuator (e.g. a hand or gripper). 
\end{myExample}

In the example above, the information on the letters themselves provides the correspondence between each letter and the location of the recipient's office. Then, during execution, as new letters are received, those groups are populated with propositions, the robot's discrete abstraction is updated, and the mission specification is translated into LTL formulas accordingly. These steps are described in the remainder of this section.

\subsection{Exploring the Physical Workspace}

In the robotics domain, more often than not, the state-space has a physical subset: the workspace that the robot operates in. When defining tasks using the open world abstractions that were introduced in Section \ref{abstractions}, one needs to account for the fact that the robot could simply restrict itself to its current workspace, never discovering new regions, without violating the specification. Therefore, part of the specification has to direct the robot to expand its physical workspace, and incorporate newly discovered regions to the discrete abstraction and mission specification.

To this end, we dedicate a portion of the mission specification to this ``exploration sub-task". It is worth noting that (i) the approach is still valid in the case that the robot already has full knowledge of its workspace, and (ii) that, since exploration is part of the specification, the user can define how the robot will deal with the open world, and under which conditions it will perform actions such as exploration or re-synthesis. The statements in this sub-task are instructions on how the robot should visit newly discovered regions (see ???).

We create the following template for defining specifications that have mixed mission and exploration goals. The specification is broken into three parts. The first part is the actual mission specification and its content does not depend on whether the robot is operating in an open world. The last part is the exploration sub-task mentioned above, and is the same for all scenarios involving exploration of the physical workspace. Therefore, it can be automatically generated, and appended to any specification. In the intermediate part, dubbed ``exploration settings", the user creates a link between the mission and exploration tasks. On one end of the spectrum, the user may choose to have the robot always be in exploration mode (e.g. \texttt{always not exploreMODE}). On the other end, the user may decide to ignore the open world aspect of the problem and restrict the mission to the currently known state-space. In between the two extremes, the user can set conditions for when the robot is allowed to explore. For example, in a search and rescue scenario, it would be sensible for the exploration settings to include the following constraint: \texttt{do exploreMODE if and only if you are not activating rescuingSurvivor}. This would prevent the robot from exploring a disaster area while attempting to help a survivor and guide him to a safe location. Note that \texttt{exploreMODE} is just a name for a binary robot proposition. Finally, the exploration settings also contain instructions for adding propositions, besides regions, such as new sensors and actions, to their corresponding groups.

\begin{algorithm}
	\textbf{Exploration sub-task:}\\
	{\small
	\texttt{...}
	}
\end{algorithm}

\subsection{Augmenting the State-Space} % Re-synthesis (not local yet)

\begin{itemize}
	\item add to / remove from
	\item BFS/DFS
	\item re-synthesis
	\item Not just LTL anymore (add to / remove from Group). First-order logic?
	\item ...
\end{itemize}

% END