%\textbf{Change section title?} % <---
\subsection{Defining Tasks over Open Worlds}

We will now leverage the power of the abstractions introduced in Section \ref{abstractions} to specify robot tasks over open worlds. We revisit Example \ref{Ex:mailbot1} and make use of the open world abstractions introduced in Section \ref{abstractions}.

\begin{myExample}\label{Ex:mailbot2} Autonomous Mailbot (revisited)\\
\texttt{Group Letters is letter1, letter2}\\
\texttt{Group Offices is office1, office2}\\
\texttt{Letters correspond to Offices}\\
\texttt{if you are sensing any Letter then do PickUp and go to corresponding Office}\\ 
\texttt{Letters} and \texttt{Offices} are groups of environment and robot (region) propositions, respectively. Notice how the task above does not refer to any specific letters (\texttt{letter1}, \texttt{letter2}) or offices (\texttt{office1}, \texttt{office2}). In fact, even if the group \texttt{Letters} was empty (\texttt{Group Letters is empty}), the user would still be able to specify the task.
\end{myExample}

If a new letter is received, .i.e., sensed by a sensor $d_{m} \in \mathcal{D}$, a new proposition, $\pi_{mk} = $\texttt{letterNew}, would be added to $AP_k$. There are two requirements for the task in Example \ref{Ex:mailbot2} to reflect the addition of the new proposition. First, \texttt{letterNew} has to be added to the group \texttt{Letters}. Second, a new office proposition, e.g. \texttt{officeNew}, has to be created and added to the group \texttt{Offices}, in order to maintain the correspondence between \texttt{Letters} and \texttt{Offices}. The two requirements define a subproblem of Problem \ref{Prob:newSpec}

\begin{myProblem}\label{Prob:correspondence}
	\textbf{(New Propositions and Correspondence):} Given a mission specification $\mathcal{M}_{\varphi [k]}$, and a new proposition $\pi_{mk} = A(m, k)$, define a mechanism for (i) adding $\pi_{mk}$ to a group of propositions $G$ in $\mathcal{M}_{\varphi [k]}$, and (ii) creating additional propositions, if necessary, and adding them to the corresponding groups $C(G)$.
\end{myProblem}

\subsection{Rewriting the Specification} % Re-synthesis (not local yet)

To address Problem \ref{Prob:correspondence}, we introduce another element of our specification language, the \texttt{add to} operator.

\begin{myDefinition}\label{Def:addto}
	\textbf{(\texttt{add to}):}
	\ldots
\end{myDefinition}

Since we are now adding propositions besides those added by the detection sensors $d_m \in \mathcal{D}$, we restate Definition \ref{Def:openworld} as follows:

\begin{myDefinition}\label{Def:openworld2}	
	\textbf{(Open World Model -- revisited):} We include the propositions added in order to maintain correspondence to the updated set of atomic propositions:
	\begin{equation}\label{Eq:updateAP2}
		AP_{k+1} = AP_k \bigcup_{m=1}^{M}A(m, k) \bigcup_{m=1}^{M}C(A(m, k)),
	\end{equation}
	and set the index $k$ to also be incremented when the specification is re-synthesized:
	\begin{equation*}
		k \leftarrow 
		\begin{cases}
			k + 1, & \text{if } \bigvee_{m=1}^{M} x_{d_m} \vee y_{\texttt{resynthesis}} \\
			k, & \text{otherwise}
		\end{cases}
	\end{equation*}
\end{myDefinition}

\begin{myAssumption}
	The correspondence, i,e., the relationship of new elements of the open world to the robot's mission, is provided either by the robot's low-level sensors, or by prompting the human user. That is, for a proposition $\pi_{mk} = A(m, k)$, added to a group of propositions, the semantics of additional propositions $C(\pi_{mk})$ can be inferred.
	% Rephrase when Problem 2 is stated. Remember, you can use Groups now!
\end{myAssumption}

For instance, in Example \ref{Ex:mailbot2}, a new letter would result in new propositions \texttt{letterNew} and\texttt{officeNew}. The information for mapping \texttt{officeNew} to a location on the robot's workspace can be inferred from the letter itself (address), which the robot can look up in a map stored in its memory. If the new office is not part of the current map, the robot could prompt the user for its location, and add it.

\subsection{Exploring the Physical Workspace}

So far we have not made a distinction among the different types of propositions that are added to $AP_k$; environment, action, and region propositions. \ldots

\textbf{Is there a distinction after all?}

BFS/DFS \ldots

%In the robotics domain, more often than not, the state-space has a physical subset: the workspace that the robot operates in. When defining tasks using the open world abstractions that were introduced in Section \ref{abstractions}, one needs to account for the fact that the robot could simply restrict itself to its current workspace, never discovering new regions, without violating the specification. Therefore, part of the specification has to direct the robot to expand its physical workspace, and incorporate newly discovered regions to the discrete abstraction and mission specification.

%\begin{myAssumption}
%	New regions are discovered when the robot visits an adjacent region. The structure of the new region becomes known to the robot at that time instant. In the future, we will remove this assumption and add a mapping aspect. \ldots
%\end{myAssumption}
%
%To this end, we dedicate a portion of the mission specification to this ``exploration sub-task". It is worth noting that (i) the approach is still valid in the case that the robot already has full knowledge of its workspace, and (ii) that, since exploration is part of the specification, the user can define how the robot will deal with the open world, and under which conditions it will perform actions such as exploration or re-synthesis. The statements in this sub-task are instructions on how the robot should visit newly discovered regions.
%
%Putting everything together, we create the following template for defining specifications that have mixed mission and exploration tasks. Specifically, the specification is broken into three parts. The first part is the actual mission specification and its content does not depend on whether the robot is operating in an open world. The last part is the exploration sub-task mentioned above, and is the same for all scenarios involving exploration of the physical workspace. Therefore, it can be automatically generated, and appended to any specification. In the intermediate part, dubbed ``exploration settings", the user creates a link between the mission and exploration tasks. On one end of the spectrum, the user may choose to have the robot always be in exploration mode (e.g. \texttt{always do exploreMODE}). On the other end, the user may decide to ignore the open world aspect of the problem and restrict the mission to the currently known state-space. In between the two extremes, the user can set conditions for when the robot is allowed to explore. For example, in a search and rescue scenario, it would be sensible for the exploration settings to include the following constraint: \texttt{do exploreMODE if and only if you are not activating rescuingSurvivor}. This would prevent the robot from exploring a disaster area while attempting to help a survivor and guide him to a safe location. Note that \texttt{exploreMODE} is just a name for a binary robot proposition. Finally, the exploration settings also contain instructions for adding non-region propositions, such as new sensors and actions, to their corresponding groups.

%\begin{algorithm}
%	\textbf{Exploration sub-task:}\\
%	{\small
%	\texttt{...}
%	}
%\end{algorithm}

% END