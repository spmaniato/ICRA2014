Paragraphs:
\begin{itemize}
	\item High-level robot mission planning
	\item Reactivity as the first step toward open world planning
  	\item Closed, fixed, static world, workspace, state-space assumption
 	\item Recent literature
   	\item Brief statement of our contributions
   	\item Paper structure
\end{itemize}

Use some ideas from \cite{open-world-sw} when talking about open world:
\begin{itemize}
	\item Software should react to changes by self-organizing its structure and self-adapting its behavior.
	\item Closed software: composed of parts that don't change during execution.
	\item Changes in the world make new components available. System can discover and bind such components dynamically while the application is executing.
	\item In practical cases, not all requirements can or should be specified upfront. Stakeholders add new requirements during execution. For example, scientists on the MARS Rover team bla bla bla.
	\item Uncertainty and partial a priori knowledge are inherent aspects of missions, not problems that we should attempt to avoid encountering.
\end{itemize}

Some references: \cite{MurrayICRA2012}, \cite{MurrayICRA2013a},  \cite{BeltaICRA2012}, \cite{Dimos2013ICRA}, \cite{Belta2013RSS}, \cite{BingxinRSS2012}.