In recent years, tools and ideas from the formal methods and hybrid system analysis have been applied to robot motion and task planning. The resulting approaches succeed in coupling low-level feedback controllers \cite{} with high-level, discrete plans \cite{}. This gave rise to a number of methodologies that translate high-level tasks to discrete and subsequently low-level continuous controllers in a correct-by-contstruction manner \cite{}.

However, the initial approaches did not account for a dynamic and possibly adversarial environment. In \cite{KGFP_TRO09}, the authors present a framework for reactive motion and task planning. The approach takes a logic formula that contains the mission specifications, which is based on a discrete abstraction of the problem, and by solving a two-player game between the robot and its environment, outputs a hybrid controller that is guaranteed to achieve the mission. Other reactive approaches have also been proposed in \cite{Wongpiromsarn2010} and more recently in \cite{Belta2013RSS}.

Reactivity was but the first step towards a more ambitious goal: open world planning. Similar to modern software \cite{open-world-sw}, robots are tasked with operating in less structured and highly uncertain environments. Thus, there is a need for planning algorithms that adapt their behavior in order to account for discrepancies between the robot's model of the world and reality. In addition, robots are now asked to incorporate new functions during execution. Examples include reconfigurable modular robots, as well as robots that learn on-the-fly \cite{SaxenaIJRR2012} or inquire online knowledge repositories \cite{rapyuta2013}. Finally, new mission requirements may be added after the robot has been deployed, as in the case of robotic search and rescue tasks \cite{MatthiasAI2010}, or future autonomous planetary exploration missions. Uncertainty and partial a priori knowledge are inherent aspects of future robot missions, not problems that we should attempt to avoid encountering.

However, in order to design provably-correct open world planners, a number of challenges have to be overcome. First, there is the assumption that a robot has full knowledge of the structure of its workspace. This issue has been tackled in \cite{MurrayICRA2012} and \cite{MurrayICRA2013a}, where the authors introduced local re-synthesis as a way to account for local topological changes in the robot's workspace. An approach that addresses the unreactive equivalent of the same problem appeared in \cite{Dimos2013ICRA}. However, the previous papers still assume that the size of the workspace is known; that only its internal structure changes. Moreover, they do not account for augmenting the mission with additional objectives, a situation which may arise if the robot can discover new regions of its workspace, or new objects of interest. The work in \cite{BingxinRSS2012} addresses possible additions to the robot's workspace without enforcing a known workspace size, but once again the solution is limited to changes in the spatial components of the state space. A true open world planner must be able to adapt to changes in other parts of the robot's world model as well, such as available sensors and actions. Finally, open-world planning introduces the challenge of enabling users to specify tasks over a world that is at least partially unknown prior to execution. The specification language and abstractions necessary for this have been approached with both semantic \cite{MatthiasICRA2012} \cite{MatthiasAI2010} and statistical \cite{TellexAAAI2011} methods, but so far no solution provides the guarantees on behavior that a formal, correct-by-construction system is capable of. 

Brief problem statement and contributions \ldots

Paper structure \ldots

% END

%Use some ideas from \cite{open-world-sw} when talking about open world:
%\begin{itemize}
%	\item Software should react to changes by self-organizing its structure and self-adapting its behavior.
%	\item Closed software: composed of parts that don't change during execution.
%	\item Changes in the world make new components available. System can discover and bind such components dynamically while the application is executing.
%	\item In practical cases, not all requirements can or should be specified upfront. Stakeholders add new requirements during execution. For example, scientists on the MARS Rover team bla bla bla (little a priori knowledge, not continuous interaction with the robot, also teleoperation slow, etc).
%	\item Uncertainty and partial a priori knowledge are inherent aspects of missions, not problems that we should attempt to avoid encountering.
%\end{itemize}