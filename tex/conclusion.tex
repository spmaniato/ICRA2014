%\addtolength{\textheight}{-22 pt}   % This command serves to balance the column lengths
                                  % on the last page of the document manually. It shortens
                                  % the textheight of the last page by a suitable amount.
                                  % This command does not take effect until the next page
                                  % so it should come on the page before the last. Make
                                  % sure that you do not shorten the textheight too much.
In this paper, we presented an approach to specifying and updating robot missions that take place in worlds open with respect to new elements, such as new objects and regions of interest. During execution, the new elements are translated into new propositions, which are automatically incorporated into the mission specification. This is possible via (i) \emph{open-world abstractions}, which enable us to specify high-level behaviors without explicitly referring to individual propositions, and (ii) the \emph{add to Group} mechanism, which systematically augments subsets of propositions with new elements. A notable advantage of our approach is that it allows the user to specify how the new elements will be incorporated into the mission, and under which conditions the updates will be reflected in the robot's controller.

The open-world model defined in this paper is strictly expanding, i.e., it only permits the addition of propositions. Thus, a future research direction is to generalize our model in order to account for the removal of elements that are no longer pertinent to the mission. 
In addition, our method of translating an updated specification to a new execution strategy is that of globally resynthesis. We will investigate \emph{local} resynthesis approaches, in the direction of \cite{MurrayICRA2012, MurrayICRA2013a}.
The ability to remove propositions, coupled with the efficiency of local resynthesis will allow us to deal with large-scale open worlds without synthesis becoming a computational bottleneck.
Furthermore, we intend to generalize and formalize the introduction of elements of first-order logic to our specification language.
%Furthermore, we will explore the use of additional first-order logic elements, in order to enhance the expressivity of our mission specifications.

On the implementation side, we are interested in a human-robot dialogue interface that would allow robots to prompt the user regarding their response to new elements in the world.
Finally, we would like to apply our approach to modular robots \cite{ModularIROS2011} with the ability to assimilate new components found in the open world, and then use them to reconfigure themselves in order to extend their functionality on-the-fly.
% END