\subsection{Groups of Propositions} 

In order to write meaningful specifications over propositions without referring to them explicitly, we borrow notions of first-order logic in $\Lambda$. 
First-order logic differs from propositional logic and LTL by its use of predicates and quantification over entities, of which our approach is only concerned with quantification. 
Two types of quantification are allowed in first-order logic: universal quantification using $\forall$ and existential quantification using $\exists$. 
\par $\Lambda$ allows quantification over sets of propositions, which we refer to as groups. 
This allows users to specify reactive behaviors in terms of groups of propositions instead of explicitly naming propositions. 
The propositions within groups invariably appear in parallel constructions when groups are invoked in a specification, therefore they must all be grammatically interchangeable within a specification. 
This interchangeability is a property of propositions of the same type (i.e. sensors, robot propositions, or regions), therefore we further constrain groups to be sets of propositions that are all of a single type. 
In summary, a group $G$ has the following property:
\begin{equation}
	G \subseteq \mathcal{X}_t\,|\,G \subseteq \mathcal{A}_t\,|\,G \subseteq \mathcal{R}_t
\end{equation}
% TODO: The sets R, A, X are defined in section II. You can remove the explanations below. ~Spyros
Where $\mathcal{X}_t \subseteq AP_t$ is the set of sensor propositions at time $t$, $\mathcal{A}_t \subseteq AP_t$ is the set of action propositions at time $t$, and $\mathcal{R}_t \subseteq AP_t$ is the set of region propositions at time $t$. 
A mission in $\Lambda$ can define any group satisfying this property.
This generalizes the work in \cite{BingxinRSS2012}, which focused only on groups of region propositions. 
Because the requirements of $\Lambda$ are agnostic to specific language implementations, the syntax of group operations, including group definition, is not fixed. 
For instance, alternative implementations of $\Lambda$ would be possible in a formal grammar like Structured English\cite{Finucane2010} and in a natural language processor like SLURP\cite{RamanRSS2013}, with different syntax in both. 

\subsection{Group Quantifiers}

For an expression in $\Lambda$ to be quantified over a group of propositions, the expression must have both a reference to the group (e.g. the name of the group) and a quantifier. 
$\Lambda$ contains three quantifiers, which we call `any', `all', and `each'. 
For quantification to be meaningful in a robot's behavior the parser $\mathcal{P}_{\Lambda}$ must be able to translate the semantics of these first-order logic operations into LTL formulas $\varphi$.
In this section we define LTL interpretations of quantification in $\Lambda$.  
To explain the semantics of these quantifiers, we will use $\bar{\varphi}$ to denote the translation of a sentence in $\Lambda$ into first-order logic. 
\par Let $\bar{\varphi}$ be the translation of and expression in $\Lambda$ into an expression containing only LTL operators and quantified groups. 
Using $G$ to denote a group, we can write the quantification of groups in $\bar{\varphi}$ as `any($G$)', `all($G$)', or `each($G$)'. 
We will also use $[y/x]\bar{\varphi}$ to denote the expression that results from replacing all occurrences of $x$ in $\bar{\varphi}$ with $y$. 
\par
The quantifier `any($G$)' in a sentence is simply translated as the logical disjunction of every proposition in the group named by $G$. 
That is, $\bar{\varphi}$ is translated into: 
\begin{equation*}
	[ t / \text{any}(G)] \bar{\varphi}
\end{equation*}
Where: 
\begin{equation*}
	t = \bigvee \limits_{\pi_i \in G} \pi_i
\end{equation*}
\par
The quantifier `all($G$)' in a sentence is translated as the logical conjunction of every proposition in the group named by $G$. 
$\bar{\varphi}$ is therefore translated into: 
\begin{equation*}
	[ t / \text{all}(G)] \bar{\varphi}
\end{equation*}
Where:
\begin{equation*}
	t = \bigwedge \limits_{\pi_i \in G} \pi_i
\end{equation*}
\par
The quantifier `each($G$)' is similar to `all($G$)', but acts with a different semantic scope. 
`each($G$)' in a sentence denotes that the entire sentence is true for each of the propositions in $G$ separately. 
$\bar{\varphi}$ is translated into:
\begin{equation*}
 	\bigwedge\limits_{\pi_i \in G} [\pi_i / \text{each}(G)] \bar{\varphi}
\end{equation*}
\par
With these abstractions, we can write compact, expressive, and intuitive specifications. 
\par
%TODO: The following example shouldn't use structured english...
\begin{myExample}\label{Ex:quantifiers} 	
	Consider a scenario where a robot is helping to manage a hotel. 
	The robot has a group of regions representing the rooms in the hotel, a group of sensors indicating whether rooms are occupied, with a proposition for every room, and actions for cleaning a room, welcoming guests, and apologizing to guests. 
	Part of the robot's specification might be expressed in English as:
	``If any rooms are occupied and not all rooms are occupied then welcome guests. 
	If all rooms are occupied then apologize to guests. 
	If none of the rooms are occupied then go to each room and clean the room.''
	Using the quantifiers in $\Lambda$, this behavior can be expressed without explicitly naming individual propositions. 
\end{myExample}

\subsection{Proposition Correspondence} 
A consequence of abstracting specifications away from individual propositions is that relationships between propositions become more difficult to denote. 
To address this difficulty, $\Lambda$ allows the definition of correspondences between propositions in groups. 
A correspondence is a mapping from individual propositions to sets of propositions that contain at most one proposition from each group defined in the specification. 
$\Lambda$ allows users to define correspondence either by explicitly mapping atomic propositions to one another or by mapping one group onto another. 
Correspondence is denoted by the function $\mathcal{C}_\pi: AP_t \rightarrow 2^{AP_t}$, which takes in an atomic proposition and returns the set of propositions that it has been defined to correspond to, where $2^{AP}$ denotes the power set of $AP$. We say that a proposition $\pi_i$ corresponds to $\pi_j$ if and only if $\pi_j \in \mathcal{C}_\pi(\pi_i)$. 
For defining correspondences between groups, we define another correspondence function $\mathcal{C}_G$, which operates on groups of propositions.
We say that a group $G_i$ corresponds to $G_j$, written as $G_j \in \mathcal{C}_G(G_i)$ if and only if the following two conditions hold:
\begin{subequations}
\begin{align}
	&\forall \pi_j \in G_j \exists \pi_i \in G_i : \pi_j \in \mathcal{C}_\pi(\pi_i) \\
	&\forall \pi_i \in G_i : \mathcal{C}_\pi(\pi_i) \cap G_j \neq \emptyset
\end{align}
\end{subequations}
Note that these conditions necessitate that $|G_i| = |G_j|$. 
In the remainder of this paper we will omit the subscripts on $\mathcal{C}$ when its usage is clear. 
\par Once correspondence has been defined in a specification, $\Lambda$ allows the use of a `corresponding' operator to implicitly refer to the relationship between individual propositions without explicitly referring to the propositions themselves. 
To use our previous notation of $\bar{\varphi}$ and $G$, the `corresponding' operator can be written as `corresponding($G$)'. 
The `corresponding' operator is different from the quantifiers described earlier because its semantics depend on the presence of another quantified group in the same sentence. 
We can demonstrate this dependency with two English sentences: ``Go to the corresponding rooms'' and ``Take all of the guests to their corresponding rooms''.
The first sentence is unclear because it is lacking the index of correspondence provided in the second sentence, namely ``all of the guests''. 
Therefore a sentence using correspondence has two relevant groups: $G_q$, the quantified group, and $G_c$, the `corresponding' group. 
The group acting as the index of correspondence may be paired with any of the available quantifiers, but the interpretation of quantified sentences using correspondence always results in conjunction over non-quantified sentences, as occurs with the `each' quantifier. 
We can now express the translation of $\bar{\varphi}$ for a sentence using correspondence as:
\begin{equation*}
	\bigwedge \limits_{\pi_i \in G_q} [\pi_i / G_q] [\mathcal{C}(\pi_i) \cap G_c / G_c] \bar{\varphi}
\end{equation*}
Note from our earlier definition of correspondence that the set intersection in this expression results in at most one element (correspondences may contain at most one element from any group). 
If the intersection is empty, meaning that a correspondence has not been properly defined, then the sentence produces a syntax error. 

%TODO: Structured english shouldn't be here either!!
\begin{myExample}\label{Ex:corresponding}
	Correspondence is useful in a scenario in which a robot performs an identical task over many sets of related propositions, such as waiting on tables at a restaurant. 
	Our specification can define a group of region propositions representing the tables in the restaurant and a group of sensor propositions representing the robot being signalled by specific tables. We can specify our desired behavior in English as follows: 
	``Calls correspond to tables. 
	If you are sensing any calls then go to the corresponding table.''
	The abstraction of propositions into groups makes the specification more compact and readable, less error-prone, and more easily extensible. 
	To specify this behavior over another pair of call and table propositions, we need only add the new propositions to their appropriate groups. 
\end{myExample}
