\subsection{Groups of Propositions} 

In order to write meaningful specifications over propositions which are not known prior to execution, we abstract propositions into groups. We can then specify the desired reactive behaviors in terms of groups of propositions and add new propositions to the appropriate group(s) when they are discovered. A proposition group is simply an ordered collection of propositions that are all of the same type (i.e. sensors, actions, or regions). Our Structured English grammar allows us to define proposition groups in a specification with sentences of the form: ``group [groupName] is [proposition1], [proposition2], [proposition3], ... , [propositionN]''. 

\subsection{Group Quantifiers} 

Unlike atomic propositions or boolean formulas of propositions, proposition groups don't have an intrinsic truth value. A proposition group used in a specification must be paired with a quantifier to produce a truth-valued logical formula. Our Structured English grammar supports three quantifiers: any, all, and each. To explain the semantics of these quantifiers, we will use $S(g)$ to denote the logical formula produced by translating a sentence with the quantifier and group name treated as a single proposition, $g$. We will also use $G$ to denote the set of propositions in a group. 
\par
The phrase ``any [groupName]'' in a sentence is simply translated as the logical disjunction of every proposition in the group named by ``[groupName]''. That is, the complete translation of the sentence is: 
\begin{equation*}
	S( \bigvee \limits_{\phi_i \in G} \phi_i )
\end{equation*}
\par
The phrase ``all [groupName]'' in a sentence is translated as the logical conjunction of every proposition in the group named by ``[groupName]''. The complete translation of the sentence is: 
\begin{equation*}
	S( \bigwedge \limits_{\phi_i \in G} \phi_i )
\end{equation*}
\par
The quantifier ``each'' is similar to ``all'', but acts with a different semantic scope. The phrase ``each [groupName]'' in a sentence denotes that the entire sentence is true for each of the propositions in ``[groupName]'' separately. The complete translation of the sentence is:
\begin{equation*}
 	\bigwedge\limits_{\phi_i \in G} S(\phi_i)
\end{equation*}
\par
For sentences that contain multiple quantifier phrases, the parser applies these translation operations in the order that the quantifiers appear in the sentence. With these abstractions, we can write compact, expressive, and intuitive specifications. 
\par
\begin{myExample}\label{Ex:quantifiers} 	
	Consider a scenario where a robot is helping to manage a hotel. The robot has a group of regions called \texttt{room} a group of sensors called \texttt{roomsOccupied} with a proposition for every room, and actions called \texttt{cleanRoom}, \texttt{welcomeGuests}, and \texttt{apologizeToGuests}. Part of the robot's specification might be: \texttt{If not any roomsOccupied then go to each room and do cleanRoom. If any roomsOccupied and not all roomsOccupied then do welcomeGuests. If all roomsOccupied then do apologizeToGuests.} 
\end{myExample}

\subsection{Proposition Correspondence} 
A consequence of abstracting specifications away from individual propositions is that relationships between propositions become more difficult to denote. To address this difficulty, our Structured English grammar allows the definition of correspondences between propositions in groups. A correspondence is a mapping from individual propositions to sets of propositions that contain at most one proposition from each group defined in the specification. We allow two sentence forms for defining correspondence: ``[propositionA1], [propositionA2], ... , [propositionAN] correspond to [propositionB1], [propositionB2], ... [propositionBN]'' and ``[GroupName1] corresponds to [GroupName2]''. In the second form, propositions are paired in the ordering of their respective groups. Once correspondence has been defined in a specification, we can use the ``corresponding'' operator to implicitly refer to the relationship between individual propositions without explicitly referring to the propositions themselves. When a sentence contains the phrase ``the corresponding [groupName]'', the parser will find the nearest previous reference to another proposition group in the sentence. The parser will then produce a copy of the sentence for each proposition in that first group, replacing occurrences of the two group names with the elements of each that correspond. The translation of the original sentence is then the conjunction of these sentences. More formally, if $S(g_1,g_2)$ represents the LTL translation of a sentence with groups $g_1$ and $g_2$ treated as single propositions, and if ``corresponding'' is used with $g_2$ and follows $g_1$ in the sentence, then the complete translation of the sentence is:
\begin{equation*}
	\bigwedge \limits_{\phi_i \in g_1} S(\phi_i, f(\phi_i) \cap g_2)
\end{equation*}
Here $f$ is the correspondence function, and $f(\phi_i)$ represents the propositions that $\phi_i$ corresponds to. 

\begin{myExample}\label{Ex:corresponding}
	Correspondence is useful in a scenario in which a robot performs an identical task over many sets of related propositions, such as delivering letters to their appropriate destinations in a building. Our specification can define a group \texttt{letter} of sensor propositions, a group \texttt{office} of region propositions, and an action \texttt{dropOffLetter}, then we can specify our desired behavior as follows: \texttt{Letter corresponds to office. If you are sensing any letter then go to the corresponding office and do dropOffLetter.} The abstraction of propositions into groups makes the specification more compact and readable, less error-prone, and more easily extensible. To specify this delivery behavior over another pair of letter and office propositions, we need only add the new propositions to their appropriate groups and they will automatically become included in the reactive behavior. 
\end{myExample}

%\subsection{Proposition Exclusion} 
%
%...
%
%\begin{myExample}\label{Ex:other}
%	\begin{itemize}
%		\item "If you're activating any Actions, then do not any other Actions" (e.g. If you are activating 			guideSurvivor, do not any other Actions)
%		\item "If you're activating any Actions, then do all other Actions"
%		\item "If you're in any Regions, then visit each other Regions"
%	\end{itemize}
%\end{myExample}