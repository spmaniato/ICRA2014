This section provides the background information on which the remainder of the paper will build on. First, we present an overview of the blocks that comprise the controller synthesis process (discrete abstraction, logic formulas, synthesis). Then, we briefly present the aspects of the LTLMoP (Linear Temporal Logic MissiOn Planning) toolkit that are pertinent to this paper.

\subsection{Controller Synthesis}\label{preliminariesA}

\begin{itemize}
	\item discrete abstraction (region, action, environment propositions). The union of $\mathcal{X}$ and $\mathcal{Y}$, the sets of environment and robot propositions respectively, defines the state-space $\mathcal{S}$. $\Sigma$ set of valuations (True/False) of the atomic propositions $\pi \in AP$.
	\item linear temporal logic (super-short)
	\item reactivity: GR(1) assume-guarantee, 6 sub-formulas
	\item synthesis, discrete strategy
	\item hybrid controller
\end{itemize}

\subsection{LTL Mission Planning}
% TODO: Rename section to "LTL from Structured English" ?

The Linear Temporal Logic Mission Planner (LTLMoP) \cite{Finucane2010} is a python-based, open-source toolkit for controlling physical and simulated robots using high-level behavior specifications. LTLMoP allows users to specify missions in either pure LTL, natural language via a module called SLURP, or a formal grammar called Structured English. Structured English is the most commonly used specification language and will be used for demonstration in this paper. Specifications in Structured English are parsed using a feature-based context-free grammar and translated into formulas of LTL for controller synthesis. 