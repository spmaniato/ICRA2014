%Using the terminology in Section \ref{preliminaries}, we can express the notion of a \emph{closed} world. Specifically, in our framework, the world is closed if and only if $AP$, the set of atomic propositions from which LTL specifications are constructed, is fixed. 
Using the terminology in Section \ref{preliminaries}, we say that the world is \emph{closed} if and only if $AP$, the set of atomic propositions, is fixed. 
That is, if the robot encounters a new element in its world (such as a new object), which was not modeled as a proposition $\pi \in AP$, it will essentially ignore it, even if it is relevant to the mission.

Three main reasons motivate the need for open-world mission specification and planning. First, in applications such as autonomous search and rescue scenarios, the mission is often specified before the robot has obtained full knowledge of the world. Second, the robot may have to incorporate new functions and respond to new events during execution. Finally, the mission's set of atomic propositions $AP$ may be too large for the user to enumerate a priori.
%To clarify these reasons, we introduce an illustrative example:

We begin our formal analysis of the problem by defining a model of an open world. Let $\W$ be a sequence of sets of atomic propositions,
%\begin{equation}\label{Eq:W}
%	\W = AP_0 AP_1 \ldots AP_k,
%\end{equation}
$\W = AP_0 AP_1 \ldots AP_k$, where $AP_{k+1} \not = AP_k$, and $k \in \NN$.

Also, let $\mathcal{D} = \left\{ d_1, \ldots, d_m, \ldots, d_M \right\}$ represent the abstract sensors that are responsible for detecting $M$ different \emph{classes} of new elements in the open world, where $\mathcal{D} \subseteq \mathcal{X}_k \subset AP_k, \forall k$. For instance, one sensor $d$ may detect new letters, whereas another new offices. 
The number of new elements that can be detected is not bounded by $M$, since each sensor in $\mathcal{D}$ can trigger multiple times over the course of a mission.
In addition, let $D$ be a function that reads the sensors abstracted by $\mathcal{D}$, and returns a new proposition if the sensors detected an unmodeled element of the world:
\begin{equation}\label{Eq:detection}
	D(m, t) = 
	\begin{cases}
		\left\{ \pi_{mt} \right\}, & \text{if } d_m[t] = \texttt{True} \\
		\emptyset, &  \text{if } d_m[t] = \texttt{False}
	\end{cases}
\end{equation}�
where $d_m[t]$ denotes the value of the binary sensor $d_m$ at position $t$ of the infinite sequence $\sigma$ (Section \ref{preliminariesA}), and $\pi_{mt} \not \in AP_t$. The new proposition $\pi_{mt}$ can be either a sensor, an action, or a region proposition.

Putting everything together, we state the open-world model that we consider in this paper.

\begin{myDefinition}\label{Def:openworld}	
	\textbf{(Expanding Open World Model):}
	The sequence $\W$ is an expanding open world model if the set of atomic propositions $AP$ is updated as follows:
	\begin{equation}\label{Eq:updateAP}
		AP_{t+1} = AP_t \cup \bigcup_{m=1}^{M}D(m, t),
	\end{equation}
	and if $AP$ changes, $AP_{t+1}$ is appended to $\W$:
	\begin{equation}\label{Eq:updateW}
	\W = 
	\begin{cases}
		\W \circ AP_{t+1}, & \text{if } \bigvee_{m=1}^{M} (d_m[t] = \texttt{True}) \\
		\W, & \text{otherwise}
	\end{cases}
	\end{equation}
As a corollary, if the world is closed, then $\W = AP_0, \forall t$.
%w.r.t. elements sensed by the sensors $d_m \in \mathcal{D}$

%	and the index $k$ is then incremented only if any of the $M$ detection sensors $\mathcal{D}$ become $\texttt{True}$:
%	\begin{equation*}
%		k \leftarrow 
%		\begin{cases}
%			k + 1, & \text{if } \bigvee_{m=1}^{M} x_{d_m} \\
%			k, & \text{otherwise}
%		\end{cases}
%	\end{equation*}
	\end{myDefinition} 

%We will return to this definition in Section \ref{openworld}, where we augment the update of the sets $AP_t$ in Eq. \eqref{Eq:updateAP} with additional new propositions. 
We will return to this definition in Section \ref{openworld}, where we augment the update of $AP_t$ with additional new propositions. 

Notice the assumptions regarding $\W$ that Eq. \eqref{Eq:updateAP} makes:

\begin{myAssumption}\label{Ass:noRemove}
	$\W$ only models the {\em addition} of propositions.
%	as a result of new elements of the open world. 
	That is, $AP_{k} \subset AP_{k+1}$. In other words, propositions cannot be removed. Hence, $\W$ is dubbed an \emph{expanding} open world model.
\end{myAssumption}

\begin{myAssumption}\label{Ass:onlyM}
	$\W$ models a world that is open only w.r.t. new elements of $M$ different types, i.e., those elements that can be sensed by the sensors $d_m \in \mathcal{D}$, $m \in \left\{ 1,\ldots, M \right\}$. 
	\end{myAssumption}

Assumption \ref{Ass:noRemove} is not limiting, since a proposition can be in $AP$ without being part of the robot's controller. Furthermore, the value of a proposition can be restricted by adding safety requirements to $\varphi_t^{e,s}$, if necessary. The disadvantage of not removing propositions that are no longer pertinent to the mission is that the synthesis algorithm \cite{piterman_06} will be searching for a winning strategy over a larger graph, which leads to a higher computation time.
Assumption \ref{Ass:onlyM} limits the new events to which the robot will react, to those that it can sense and that are relevant to the mission. For instance, the mailbot (Example \ref{Ex:mailbot1}) should update its mission if it senses a new recipient, but not if someone tries to hand it a trash bag.

In order to obtain a mission specification and a robot plan that react to unmodeled elements of the world, we need to address two main problems. First, we should allow specifications that include operators outside of LTL, in order to enable the mission to adapt to $\W$. Then, we need to define the specific mechanism that will allow new propositions $\pi_{mt}$ to be incorporated in the mission's LTL formulas.

The problem of rewriting a specification to allow for adding new propositions is stated below:

\begin{myProblem}\label{Prob:newSpec}
	\textbf{(Mission Specification Update):} Let $\Lambda$ be a specification language and $\mathcal{P}_{\Lambda}$ a parser for it. % mention CFG?
	Augment $\Lambda$ and $\mathcal{P}_{\Lambda}$, such that given a mission specification written in $\Lambda$, $\mathcal{M}_k = \mathcal{M}(AP_k) \in \Lambda$, a GR(1) formula $\varphi [k]$, and the latest set of atomic propositions, $AP_{k+1} \supset AP_k$, the mission $\mathcal{M}_k$ and formula $\varphi_k$ are automatically updated as follows:
	\begin{subequations}\label{Eq:newSpec}
		\begin{align}
			\mathcal{M}_{k+1} &= \mathcal{M}(AP_{k+1}), \label{Eq:newSpecA}\\ 
%			\varphi [k+1] &= \mathcal{P}_{\Lambda} (\mathcal{M}_{k+1}, AP_{k+1}),
			\varphi [k+1] &= \mathcal{P}_{\Lambda} (\mathcal{M}_{k+1}), \label{Eq:newSpecB}
		\end{align}
	\end{subequations}
	where the initial conditions of the GR(1) formula $\varphi [k+1]$ are set such that $\sigma (t^+)\models \varphi_i^{e}[k+1] \wedge \varphi_i^{s}[k+1]$, where $\sigma (t^+)$ are the current truth assignments to the atomic propositions $\pi \in AP_{k+1}$ right after $AP_k$ is updated.
%	such that
%	\begin{align*} % This is NOT right! The automaton has not been updated yet...
%		x_{k+1} &\models \varphi_i^e [k+1], \; x_{k+1} \in \Sigma_{\mathcal{X}_{k+1}} = \Sigma_{\mathcal{X}_{k}} \cup \Sigma_{\mathcal{X}_{k+1} \backslash \mathcal{X}_{k}} \\
%		y_{k+1} &\models \varphi_i^s [k+1], \; y_{k+1} \in \Sigma_{\mathcal{Y}_{k+1}} = \Sigma_{\mathcal{Y}_{k}} \cup \Sigma_{\mathcal{Y}_{k+1} \backslash \mathcal{Y}_{k}}
%	\end{align*}
	% Also include a constraint on safeties and livenesses.
\end{myProblem}
These restrictions on $\varphi [k+1]$ should ensure that its initial conditions are compatible with the current environment and robot state. The initial valuations of the new propositions $\pi \in AP_{k+1} \backslash AP_{k}$ are set according to the sensor readings and the constraints in $\mathcal{M}_{k+1}$.
%Also, these restrictions set the values $x,y$ of any new propositions $\pi \in AP_{k+1} \backslash AP_{k}$. Furthermore, the notation $\mathcal{M}_{\varphi [k]}$ means that a subset of a mission specification $\mathcal{M}$ is equivalent to the GR(1) formula $\varphi[k]$. In other words $\mathcal{M}_{\varphi [k]}$ consists of a fixed part and a part that is updated when $\varphi[k]$ is updated. For brevity, we will often drop the subscript $\varphi [k]$ from $\mathcal{M}$.
Notice that the specification language $\Lambda$ and the parser $\mathcal{P}_{\Lambda}$ only provide the semantics and mechanics for updating $\varphi[k]$. The resulting formula, $\varphi [k+1]$, depends on the user-defined mission specification $\mathcal{M}$.

%Given the problem statement above, we begin our definition of the specification language $\Lambda$ with the notion of \emph{open world abstractions}. These abstractions allow us to specify tasks without explicitly referring to individual propositions. In Section \ref{abstractions} we show how they enable the systematic addition of new propositions to $AP_k$,  as well as the meaningful rewriting of $\mathcal{M}_k$ and subsequently $\varphi[k]$.
Given the problem statement above, we first augment the specification language $\Lambda$ with \emph{open world abstractions} in Section \ref{abstractions}.
These abstractions allow us to specify tasks without explicitly referring to individual propositions. 
In Section \ref{openworld} we show how an additional mechanism of $\Lambda$ leverages the expressive power of open world abstractions to enable the systematic addition of new propositions to $AP_k$,  and consequently the rewriting of $\mathcal{M}_k$ and $\varphi[k]$.

% END

%The first problem we aim to solve is a practical one. In a scenario such as that of Example \ref{Ex:mailbot1}, parts of the robot's state space (e.g. letters to carry) are potentially very large and could conceivably be expanded (or reduced) by the user in subsequent runs. To specify the same reactive behavior over an additional proposition in the domain, the user would have to modify or duplicate nearly every sentence in the specification. A more preferable approach would be to specify behaviors with an abstraction that does not explicitly refer to individual propositions.
%
%\begin{myProblem}\label{Prop:groups}
%	\textbf{(Specification Abstractions):}
%	Given a robot mission that includes indentical behavior over multiple propositions, express that mission in a specification language such that the individual propositions acted on identically are not explicitly referred to. 
%\end{myProblem}
%
%Our approach to Problem 1 is to include elements of first-order logic in our specification language, specifically set operations. We will first explain the theory and use of these abstractions in our specifications (Section \ref{abstractions}), and then apply them to the second part of our overall problem, updating an open world mission specification (Section \ref{openworld}). 