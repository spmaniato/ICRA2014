...

\begin{myDefinition}
	\textbf{(Open World):} The union of $\mathcal{X}$ and $\mathcal{Y}$, the sets of environment and robot proposition respectively, defines the state-space. If the state-space is allowed to increase and decrease in size during execution, then we say that the robot operates under the open world assumption.
\end{myDefinition}
% FIXME: Either remove "decrease" from the definition above, or comment on it afterwards.
In real-world applications, such as search and rescue scenarios, there is a need to define a mission before the robot has obtained full knowledge of the world. In other cases, the state-space may be too large for (i) the user to enumerate every possible variation upfront, and (ii) the synthesis algorithm to deal with. To clarify the latter cases, consider the following mission:

\begin{myExample}\label{Ex:mailman1} Robotic Mailman\\
	A robotic mailman operates within a school building. It is tasked with collecting letters and delivering them to the offices of the corresponding recipients. \ldots
\end{myExample}

The example above demonstrates \ldots

\begin{myProblem}
	\textbf{(Open World Mission Re-Planning):} \ldots
\end{myProblem}

We are not assuming that the robot's physical workspace is of some unknown fixed size. We do, however, make the following assumptions (TODO: either in this section, or the one on Open World Synthesis) \ldots

\begin{myAssumption}
	New regions are discovered when the robot visits an adjacent region. The structure of the new region becomes known to the robot at at time instant. In the future, we will remove this assumption and add a mapping aspect. \ldots
\end{myAssumption}

\begin{myAssumption}
	\ldots
\end{myAssumption}