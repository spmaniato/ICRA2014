%%%%%%%%%%%%%%%%%%%%%%%%%%%%%%%%%%%%%%%%%%%%%%%%%%%%%

\documentclass[letterpaper, 10 pt, conference]{ieeeconf}  % Comment this line out if you need a4paper

\IEEEoverridecommandlockouts                        	% This command is only needed if 
                                                         	 		% you want to use the \thanks command
\overrideIEEEmargins                                      	% Needed to meet printer requirements.

% Packages:
\usepackage{graphics} 		% for pdf, bitmapped graphics files
\usepackage{epsfig} 			% for postscript graphics files
%\usepackage{mathptmx} 	% assumes new font selection scheme installed
\usepackage{times} 			% assumes new font selection scheme installed
\usepackage{amsmath} 		% assumes amsmath package installed
%\usepackage{amsthm}
\usepackage{amssymb}  		% assumes amsmath package installed
\usepackage{listings}
\usepackage{stmaryrd}
\usepackage{graphicx, float, wrapfig}
%\usepackage{subfigure}
\usepackage{subfig}
\usepackage{tikz}
%\usepackage{algpseudocode}
\usepackage{algorithm}
\usepackage{algorithmic}
\usepackage{mathtools}
\usepackage{cite}
\usepackage[font = small]{caption}

% New commands:
\newcommand{\empha}[1]{ \textbf{\emph{#1}}}
\DeclareMathOperator{\F}{\rotatebox[origin=c]{45}{$\Box$}}
\DeclareMathOperator{\X}{\bigcirc}
\DeclareMathOperator{\G}{\Box}
\newcommand{\LTLG}{\G}
\newcommand{\LTLF}{\F}
\newcommand{\LTLX}{\X}
%\newcommand{\NN}{\mathbb{N}}
%\newcommand{\RR}{\mathbb{R}}
%\DeclareMathOperator{\Cox}{\scalebox{0.8}{$\F$}\hspace{-9.3pt}\bigcirc}
\DeclareMathOperator{\Cox}{
    \begin{tikzpicture}[baseline=(X.base)]
    \path[use as bounding box] (-0.16,-0.17) -- (0.16,0.17);
    \node at (0,0) {\scalebox{0.8}{$\F$}}; \node (X) at(0,0) go{$\bigcirc$}; 
    %\draw (current bounding box.north west) rectangle (current bounding box.south east);
    \end{tikzpicture}}
    
% Define an "Example" environment with its own counter    
\newcounter{examplecounter}		% Example counter. Independent of theorem and other counters.
\newenvironment{myExample}
{
	\refstepcounter{examplecounter}%
	\textbf{Example \arabic{examplecounter}:}% Example #.
	\quad							% Let's give our text some space :-)
}

% Define a "Definition" environment with its own counter    
\newcounter{definitioncounter}		% Definition counter. Independent of theorem and other counters.
\newenvironment{myDefinition}
{
	\refstepcounter{definitioncounter}%
	\textbf{Definition \arabic{definitioncounter}}%
}

% Define a "Problem" environment with its own counter    
\newcounter{problemcounter}
\newenvironment{myProblem}
{
	\refstepcounter{problemcounter}%
	\textbf{Problem \arabic{problemcounter}}% Definition #.
}

% Define an "Assumption" environment with its own counter    
\newcounter{assumptioncounter}
\newenvironment{myAssumption}
{
	\refstepcounter{assumptioncounter}%
	\textbf{Assumption \arabic{assumptioncounter}:}%
}
    
%%%%%%%%%%%%%%%%%%%%%%%%%%%%%%%%%%%%%%%%%%%%%%%%%%%%%

\title{\LARGE \bf
	High-level Mission Specification and Planning for Open World Robots
}

\author{Spyros Maniatopoulos, Matthew Blair and Hadas Kress-Gazit% 	<-this % stops a space
%\thanks{*This work was not supported by any organization}% 			<-this % stops a space
\thanks{The authors are with the Autonomous Systems Lab, Sibley School of Mechanical and Aerospace Engineering, Cornell University, Ithaca, NY 14853, USA, {\tt \{sm2296, meb286, hadaskg\}@cornell.edu}}%
%\thanks{$^{2}$Bernard D. Researcheris with the Department of Electrical Engineering, Wright State University,
%        Dayton, OH 45435, USA
 %     {\tt\small b.d.researcher@ieee.org}}%
}

%%%%%%%%%%%%%%%%%%%%%%%%%%%%%%%%%%%%%%%%%%%%%%%%%%%%%

\begin{document}

\maketitle
\thispagestyle{empty}
\pagestyle{empty}

%%%%%%%%%%%%%%%%%%%%%%%%%%%%%%%%%%%%%%%%%%%%%%%%%%%%%

\begin{abstract}

% Check LTL in LaTeX: $\X \G \F \Cox$.
...

\end{abstract}

%%%%%%%%%%%%%%%%%%%%%%%%%%%%%%%%%%%%%%%%%%%%%%%%%%%%%

\section{INTRODUCTION}
...

\begin{itemize}
	\item High-level robot mission planning
  	\item Fixed workspace (state-space) assumption
 	\item Recent literature
   	\item Brief statement of our contributions
   	\item Paper structure
\end{itemize}

Some references: \cite{MurrayICRA2012}, \cite{MurrayICRA2013a},  \cite{BeltaICRA2012}, \cite{Dimos2013ICRA}, \cite{Belta2013RSS}, \cite{BingxinRSS2012}.

\section{PRELIMINARIES}\label{preliminaries}
This section provides the background information on which the remainder of the paper will build on. First, we present an overview of the blocks that comprise the controller synthesis process (discrete abstraction, logic formulas, synthesis). Then, we briefly present the aspects of the LTLMoP (Linear Temporal Logic MissiOn Planning) toolkit that are pertinent to this paper.

\subsection{Controller Synthesis}

\begin{itemize}
	\item discrete abstraction (region, action, environment propositions). The union of $\mathcal{X}$ and $\mathcal{Y}$, the sets of environment and robot propositions respectively, defines the state-space $\mathcal{S}$. $\sigma(\pi)$ is an evaluation of the atomic proposition $\pi$?
	\item linear temporal logic (super-short)
	\item reactivity: GR(1) assume-guarantee, 6 sub-formulas
	\item synthesis, discrete strategy
	\item hybrid controller
\end{itemize}

\subsection{LTL Mission Planning}
% TODO: Rename section to "LTL from Structured English" ?

The Linear Temporal Logic Mission Planner (LTLMoP) \cite{Finucane2010} is a python-based, open-source toolkit for controlling physical and simulated robots using high-level behavior specifications. LTLMoP allows users to specify missions in either pure LTL, natural language via a module called SLURP, or a formal grammar called Structured English. Structured English is the most commonly used specification language and will be used for demonstration in this paper. Specifications in Structured English are parsed using a feature-based context-free grammar and translated into formulas of LTL for controller synthesis. 

\section{PROBLEM SETUP}\label{problem}
\subsection{Motivation}

Using the terminology in Section \ref{preliminaries}, we can express the notion of a \emph{closed} world in the current context. Specifically, the closed world assumption implies that $AP$, the set of atomic propositions with which LTL specifications are constructed from, is fixed. That is, if the robot encounters a new element of its world, which is not modeled as a proposition $\pi \in AP$, it will essentially ignore it, even if it is relevant to the mission.

Three main reasons motivate the need for open world mission specification and planning. First of all, in real-world applications, such as autonomous search and rescue scenarios, the mission is often specified before the robot has obtained full knowledge of the world. Second, the robot may have to incorporate new function and components to its strategy. Finally, the state-space may be too large for the user to enumerate every possible variation a priori.  To clarify these reasons, we introduce an illustrative example:

\begin{myExample}\label{Ex:mailbot1} Autonomous Mailbot (Fig. \ref{Fig:pr3})\\
	A robotic courier (mailbot) operates within a school or company building. It is tasked with collecting letters and delivering them to the recipients' offices. Even if all possible recipients, and the locations of their offices, are fixed, the information may not be available at the time the mission specification is defined.
\end{myExample}

\begin{figure}[ht]
	\centering
	\includegraphics[width=0.7\columnwidth, clip]{./img/pr3.jpg}
	\caption{Our implementation of a mailbot. A Nao humanoid robot (ref ???) is mounted on a segway platform. The actuation and perception capabilities of the former are complemented by the localization, navigation and mobility advantages of the latter.}
	\label{Fig:pr3}
\end{figure}

In Example \ref{Ex:mailbot1}, notice that the robot does not operate serially, i.e., collecting one letter, delivering it, and returning for the next one. Rather, it is allowed to carry multiple letters at once, collect letters on its way to delivering a different letter, etc. Therefore, the specification must make sure that the robot only delivers letters at the correct location, that it is aware of which letters it is carrying, and that it does not forget to deliver a letter altogether. The open world aspect of the mission comes from the fact that new letters that correspond to new recipients may be collected by the robot during execution. Therefore, it is not possible to explicitly write a specification in the form of individual delivery tasks, e.g., \texttt{if you are sensing letter\_John\_Doe then do PickUp\_Letter and go to John\_Does\_office}. Notice that the detection of a letter addressed to a new recipient necessitates both its modeling, as a proposition, and the addition of another objective in the specification.

\subsection{Problem Statement}

We begin our formal analysis of the problem by defining a model of the open world, compatible with the current setting; that of controller synthesis.

Let $\W$ be a sequence of sets of atomic propositions:
\begin{equation}\label{Eq:W}
	\W = \left\{ AP_0, AP_1, \ldots, AP_k \right\}, \quad k \in \NN,
\end{equation} 
where $\mathcal{D} = \left\{ d_1, \ldots, d_m, \ldots, d_M \right\} \subseteq \mathcal{X}_k \subset AP_k, \forall k$. The abstract sensors in $\mathcal{D}$ are responsible for detecting $M$ types of new elements of the open world. For instance, one sensor $d$ may detect new letters, whereas another new offices.

In addition, let $A$ be a function which reads the sensors that environment propositions $\mathcal{D}$ abstract, and returns a new proposition if they detected an unmodeled element of the world:
\begin{equation}\label{Eq:A}
	A(m, k) = 
	\begin{cases}
		\left\{ \pi_{mk} \right\}, & \text{if } x_{d_m} \\ % Refers to valuation {0,1}
		\left\{ \right\}, &  \text{if } \neg x_{d_m}
	\end{cases}
\end{equation}�
where $x_{d_m} \in \Sigma_\mathcal{D}$ is the valuation of the $m$-th detection sensor, and $\pi_{mk} \not \in AP_k$. The new proposition $\pi_{mk}$ can be either an environment, or an action, or a region proposition.

Putting together Equations \eqref{Eq:W} -- \eqref{Eq:A}, we state our interpretation of an open world model.

\begin{myDefinition}\label{Def:openworld}	
	\textbf{(Open World Model):}
	The sequence $\W$ is an open world model if the sets $AP_k$ are iteratively defined:
	\begin{equation}\label{Eq:updateAP}
		AP_{k+1} = AP_k \bigcup_{m=1}^{M}A(m, k),
	\end{equation}
	and the index $k$ is then updated only if any of the $M$ detection sensors $\mathcal{D}$ become $\texttt{True}$:
	\begin{equation*}
		k \leftarrow 
		\begin{cases}
			k + 1, & \text{if } \bigvee_{m=1}^{M} x_{d_m} \\
			k, & \text{otherwise}
		\end{cases}
	\end{equation*}
	\end{myDefinition} 

We will return to this definition in Section \ref{openworld}, where we augment the update of the sets $AP_k$ in Eq. \eqref{Eq:updateAP} with additional new propositions.

In order to obtain a mission specification and a robot plan that can react to unmodeled elements of the world, we need to address two main problems. First of all, we should allow specifications that include formulas outside LTL, in order to allow ``self-adapting" missions (Problem \ref{Prob:newSpec}). Then, we need to define the specific mechanics of the new specification, which will allow new propositions $\pi_{mk} \in AP_{k+1}$ to be incorporated in LTL formulas (Sections \ref{abstractions}, \ref{openworld}).

The problem of rewriting a specification to allow for adding new propositions is formally stated below:

\begin{myProblem}\label{Prob:newSpec}
	\textbf{(Mission Specification Update):} Define a specification language $\Lambda$, and a parser $\mathcal{P}$ for it. Then, given a mission specification $\mathcal{M}$ written in $\Lambda$, a GR(1) formula $\varphi [k]$, and the latest set of atomic propositions $AP_{k+1}$, the LTL specification is updated:
	\begin{equation}\label{Eq:newSpec}
		\varphi [k+1] = \mathcal{P} (\mathcal{M}_{\varphi [k]}, AP_{k+1}),
	\end{equation}
	such that
	\begin{align*} % This is NOT right! The automaton has not been updated yet...
		x_{k+1} &\models \varphi_i^e [k+1], \; x_{k+1} \in \Sigma_{\mathcal{X}_{k+1}} = \Sigma_{\mathcal{X}_{k}} \cup \Sigma_{\mathcal{X}_{k+1} \backslash \mathcal{X}_{k}} \\
		y_{k+1} &\models \varphi_i^s [k+1], \; y_{k+1} \in \Sigma_{\mathcal{Y}_{k+1}} = \Sigma_{\mathcal{Y}_{k}} \cup \Sigma_{\mathcal{Y}_{k+1} \backslash \mathcal{Y}_{k}}
	\end{align*}
	% Also include a constraint on safeties and livenesses.
\end{myProblem}

The restrictions on $\varphi [k+1]$, the updated GR(1) specification, ensure that its initial conditions are compatible with the current ($k$) environment and robot state. Also, these restrictions set the values $x,y$ of any new propositions $\pi \in AP_{k+1} \backslash AP_{k}$. Furthermore, the notation $\mathcal{M}_{\varphi [k]}$ means that a subset of a mission specification $\mathcal{M}$ is equivalent to the GR(1) formula $\varphi[k]$. In other words $\mathcal{M}_{\varphi [k]}$ consists of a fixed part and a part that is updated when $\varphi[k]$ is updated.

Notice that the specification language $\Lambda$ and the parser $\mathcal{P}$ only provide the semantics and mechanics for updating $\varphi[k]$. The resulting formula, $\varphi [k+1]$, depends on the settings of the user-defined mission specification $\mathcal{M}$.

Given the problem statement above, we begin our definition of the new specification language $\Lambda$ with the notion of \emph{open world abstractions}. These abstractions allow us to specify tasks without explicitly referring to individual propositions. In Section \ref{abstractions} we show how they enable the systematic addition of new propositions to $AP_k$,  as well as the meaningful rewriting of $\varphi[k]$.

% END

%The first problem we aim to solve is a practical one. In a scenario such as that of Example \ref{Ex:mailbot1}, parts of the robot's state space (e.g. letters to carry) are potentially very large and could conceivably be expanded (or reduced) by the user in subsequent runs. To specify the same reactive behavior over an additional proposition in the domain, the user would have to modify or duplicate nearly every sentence in the specification. A more preferable approach would be to specify behaviors with an abstraction that does not explicitly refer to individual propositions.
%
%\begin{myProblem}\label{Prop:groups}
%	\textbf{(Specification Abstractions):}
%	Given a robot mission that includes indentical behavior over multiple propositions, express that mission in a specification language such that the individual propositions acted on identically are not explicitly referred to. 
%\end{myProblem}
%
%Our approach to Problem 1 is to include elements of first-order logic in our specification language, specifically set operations. We will first explain the theory and use of these abstractions in our specifications (Section \ref{abstractions}), and then apply them to the second part of our overall problem, updating an open world mission specification (Section \ref{openworld}). 

\section{SET ABSTRACTIONS}\label{abstractions}
\subsection{Groups of Propositions} 

In order to write meaningful specifications over propositions without referring to them explicitly, we allow our specification language $\Lambda$ to abstract propositions into groups. 
This allows us to specify our desired reactive behaviors in terms of groups of propositions and add new propositions to the appropriate group(s) when needed, making use of the existing logic in the specification. 
A proposition group is simply an ordered collection of propositions that are all of the same type (i.e. sensors, actions, or regions). 
Our Structured English grammar allows us to define proposition groups in a specification with sentences of the form: 
``group \textit{groupName} is \textit{proposition1}, \textit{proposition2}, \textit{proposition3}, ... , \textit{propositionN}''. 
This generalizes the work in \cite{BingxinRSS2012} focusing on groups of region propositions. 
Note that this syntax is just one possible implementation of the abstractions we describe. 
An alternative implementation would be possible in a natural language processor like SLURP \cite{RamanRSS2013}, for instance. 

\subsection{Group Quantifiers} 

Unlike atomic propositions or boolean formulas of propositions, proposition groups don't have an intrinsic truth value. 
A proposition group used in a specification must be paired with a quantifier to produce a truth-valued logical formula. 
Our Structured English grammar supports three quantifiers: any, all, and each. 
To explain the semantics of these quantifiers, we will use $S(g)$ to denote the logical formula produced by translating a sentence with a quantifier and group name treated as a single proposition, $g$. 
We will also use $G$ to denote the set of propositions in a group. 
\par
The phrase ``any \textit{groupName}'' in a sentence is simply translated as the logical disjunction of every proposition in the group named by \textit{groupName}. 
That is, the complete translation of the sentence $S$ is: 
\begin{equation*}
	S( \bigvee \limits_{\phi_i \in G} \phi_i )
\end{equation*}
\par
The phrase ``all \textit{groupName}'' in a sentence is translated as the logical conjunction of every proposition in the group named by \textit{groupName}. 
The complete translation of the sentence $S$ is: 
\begin{equation*}
	S( \bigwedge \limits_{\phi_i \in G} \phi_i )
\end{equation*}
\par
The quantifier ``each'' is similar to ``all'', but acts with a different semantic scope. 
The phrase ``each \textit{groupName}'' in a sentence denotes that the entire sentence is true for each of the propositions in \textit{groupName} separately. 
The complete translation of the sentence $S$ is:
\begin{equation*}
 	\bigwedge\limits_{\phi_i \in G} S(\phi_i)
\end{equation*}
\par
For sentences that contain multiple quantifier phrases, the parser applies these translation operations in the order that the quantifiers appear in the sentence. 
With these abstractions, we can write compact, expressive, and intuitive specifications. 
\par
\begin{myExample}\label{Ex:quantifiers} 	
	Consider a scenario where a robot is helping to manage a hotel. 
	The robot has a group of regions called \texttt{room} a group of sensors called \texttt{roomsOccupied} with a proposition for every room, and actions called \texttt{cleanRoom}, \texttt{welcomeGuests}, and \texttt{apologizeToGuests}. 
	Part of the robot's specification might be: 
	\texttt{If any roomsOccupied and not all roomsOccupied then do welcomeGuests. 
	If all roomsOccupied then do apologizeToGuests. 
	If not any roomsOccupied then go to each room and do cleanRoom. } 
\end{myExample}

\subsection{Proposition Correspondence} 
A consequence of abstracting specifications away from individual propositions is that relationships between propositions become more difficult to denote. 
To address this difficulty, our Structured English grammar allows the definition of correspondences between propositions in groups. 
A correspondence is a mapping from individual propositions to sets of propositions that contain at most one proposition from each group defined in the specification. 
We allow two sentence forms for defining correspondence: 
\begin{itemize}
	\item ``\textit{PropA1}, \textit{PropA2}, ... , \textit{PropAN} correspond to \textit{PropB1}, \textit{PropB2}, ... , \textit{PropBN}''
	\item ``\textit{GroupName1} corresponds to \textit{GroupName2}''
\end{itemize} 
In the second form, propositions are paired in the ordering of their respective groups. 
Once correspondence has been defined in a specification, we can use the ``corresponding'' operator to implicitly refer to the relationship between individual propositions without explicitly referring to the propositions themselves. 
When a sentence contains the phrase ``the corresponding \textit{groupName}'', the parser will find the nearest previous reference to another proposition group in the sentence. 
The parser will then produce a copy of the sentence for each proposition in that first group, replacing occurrences of the two group names with the elements of each that correspond. 
The translation of the original sentence is then the conjunction of these sentences. 
We can state this more formally: Let $S(g_1,g_2)$ represent the LTL translation of a sentence with quantified groups $g_1$ and $g_2$ treated as single propositions. 
If ``corresponding'' is used with $g_2$, then the complete translation of the sentence is:
\begin{equation*}
	\bigwedge \limits_{\phi_i \in G_1} S(\phi_i, C(\phi_i) \cap G_2)
\end{equation*}
Where $G_1$ and $G_2$ are the sets of propositions in groups $g_1$ and $g_2$ respectively.
$C$ is the correspondence function such that $C(\phi_i)$ represents the set of propositions that $\phi_i$ corresponds to. 
In our implementation, $C$ is a dictionary with atomic propositions as the keys and lists of corresponding propositions as the values. 
Note from our earlier definition of correspondence that the set intersection in this expression results in at most one element (correspondences may contain at most one element from any group). 
If the intersection is empty, meaning that a correspondence has not been properly defined, then the sentence produces a syntax error. 

\begin{myExample}\label{Ex:corresponding}
	Correspondence is useful in a scenario in which a robot performs an identical task over many sets of related propositions, such as waiting on tables at a restaurant. 
	Our specification can define a group \texttt{tables} of region propositions, representing the tables in a restaurant, and a group \texttt{calls} of sensor propositions, representing the robot being signalled by specific tables. We can specify our desired behavior as follows: 
	\texttt{Calls correspond to tables. 
	If you are sensing any calls then go to the corresponding tables.} 
	The abstraction of propositions into groups makes the specification more compact and readable, less error-prone, and more easily extensible. 
	To specify this behavior over another pair of call and table propositions, we need only add the new propositions to their appropriate groups and they will automatically become included in the reactive behavior. 
\end{myExample}


\section{SYNTHESIS FOR OPEN--WORLDS}\label{openworld}
\subsection{Defining Tasks over Open Worlds}

First of all, let us define the notion of open world in the current context.

\begin{myDefinition}
	\textbf{(Open World).} The union of $\mathcal{X}$ and $\mathcal{Y}$, the sets of environment and robot proposition respectively, defines the state-space. If the state-space is allowed to increase in size during execution, then we say that the robot operates under the open world assumption.
\end{myDefinition}

In real-world applications, such as search and rescue scenarios, there is a need to define a mission before the robot has obtained full knowledge of the world. In other cases, the state-space may be too large for (i) the user to enumerate every possible variation upfront, and (ii) the synthesis algorithm to deal with. To clarify the latter cases, consider the following mission:

\begin{myExample}\label{Ex:mission1}
	A robotic mailman operates within a school building. It is tasked with collecting letters from a mailroom, and delivering them to the offices of the corresponding recipients. However, the robot's mission specification does not have to account for every possible recipient. Using the open world abstractions introduced in Section \ref{abstractions}, we can describe one of the robot's objective, in Structured English, as follows: 
\texttt{if you are sensing any Letter then do PickUp and go to corresponding Office}, where \texttt{Letter} and \texttt{Office} are groups of environment and location (robot) propositions, respectively.
%\texttt{each CarryLetter is set on PickupLetter and the corresponding Letter and reset on DeliverLetter}, where \texttt{CarryLetter} and \texttt{Letter} are groups of robot and environment propositions, respectively, and \texttt{PickupLetter} is a robot proposition that abstracts an actuator (e.g. a hand or gripper). 
The information on the letters themselves provides the correspondence between each letter and the location of the recipient's office. Then, during execution, as new letters are received, those groups are populated with propositions, the robot's discrete abstraction is updated, and the mission specification is translated into LTL formulas accordingly.
\end{myExample}

...

\subsection{Exploring the Physical Workspace}

In the robotics domain, more often than not, the state-space has a physical subset: the workspace that the robot operates in. When defining tasks using the open world abstractions that were introduced in Section \ref{abstractions}, one needs to account for the fact that the robot could simply choose to ignore those tasks. In other worlds, there has to be a mechanism that forces the robot to expand its physical workspace, in order to discover aspects of its mission specification.

We dedicate a portion of the mission specification to this ``exploration" task. It is worth noting that (i) the approach is still valid in the case that the robot has full knowledge of its workspace, but not its entire state-space, and (ii) that, since exploration is part of the specification, the user can define how the robot will deal with the open world, and under which conditions it will perform actions such as exploration or re-planning.

... frontier, BFS/DFS, memory, ...

\subsection{Augmenting the State-Space} % Re-synthesis (not local yet)

\begin{itemize}
	\item add to / remove from
	\item re-synthesis
	\item Not just LTL anymore (add to / remove from Group). First-order logic?
	\item ...
\end{itemize}

\section{SIMULATION: Open--World Exploration}\label{simulation}  % or make exploration a subsection in pref section
...

\begin{myExample}\label{Ex:mailman3} Mailman (revisited)
	Give all details of the spec ... \textbf{Same as demo or multiple letters but other simplifications?}
\end{myExample}

\begin{algorithm}
	\textbf{Mission specification:} Robotic Mailman\\
	{\small
	\texttt{Group Letter is empty}\\
	\texttt{Group CarryingLetter is empty}\\
	\texttt{Group Office is empty}\\
	\texttt{CarryingLetter correspond to Letter}\\
	\texttt{CarryingLetter correspond to Office}\\
	\texttt{each CarryingLetter is set on PickUpLetter and the corresponding Letter and reset on 			DeliverLetter or ReturnLetter}\\
	
	\texttt{if you are not activating any CarryingLetter then "patrol"?.}\\
	\texttt{...}
	}
\end{algorithm}

\begin{myExample}\label{Ex:planetxplore} Autonomous Planetary Exploration\\
	Consider now the more futuristic scenario of autonomous planetary exploration using rovers. In order to lessen the dependency of the rover from the engineers on Earth, the rover will be given a mission specification that it should carry out autonomously. However, the need to redefine the mission will arise as the rover discovers interesting elements of its environment, such as new types of rock and regions interest. In these cases, the engineers and scientists on Earth should be able to extend the rover's mission specification without sacrificing correctness or rewriting the specification from scratch.
	
	Specifics: exploration, new stuff sensor, new requests.
\end{myExample}

\begin{algorithm}
	\textbf{Mission specification:} Autonomous Planetary Exploration\\
	...
\end{algorithm}

\section{CONCLUSIONS AND FUTURE WORK}\label{conclusion}
Discussion:
\begin{itemize}
	\item Summary
	\item As demonstrated in Example \ref{Ex:SnS}, the rewriting rules and the time of re-synthesis are defined in the spec; by the user (unlike lit).
	\item \ldots
\end{itemize}

Future work:
\begin{itemize}
	\item Open World Model allowed to contract, not only expand.
	\item Local/Incremental re-synthesis
	\item Prompt user for feedback re newly discovered propositions via a dialogue interface.
	\item First-order logic: Set manipulations during execution, specs like ``do action iff Group is empty", \ldots
	\item Modular robots can create new actuators (action propositions)
\end{itemize}�

%\addtolength{\textheight}{-12cm}   % This command serves to balance the column lengths
                                  % on the last page of the document manually. It shortens
                                  % the textheight of the last page by a suitable amount.
                                  % This command does not take effect until the next page
                                  % so it should come on the page before the last. Make
                                  % sure that you do not shorten the textheight too much.

%%%%%%%%%%%%%%%%%%%%%%%%%%%%%%%%%%%%%%%%%%%%%%%%%%%%%

\section*{APPENDIX}

...

\section*{ACKNOWLEDGMENT}

... Cameron ...

%%%%%%%%%%%%%%%%%%%%%%%%%%%%%%%%%%%%%%%%%%%%%%%%%%%%%

\bibliographystyle{ieeetran}
\bibliography{spyros-biblio.bib}

\end{document}
